Analysis of these measures across performers is ongoing, with many patterns to be discovered. The orchestra is an exceptional example of human coordination, with many bodies simultaneously producing sound with great control and replicability. In the next months, we will carry out different analyses on the performers' data that will show us how they coordinate with the others in the orchestra and how musicians' bodies engage with the music. For example: 
\begin{itemize}
\item \textbf{Performance variability}: We will look at when in the concerts the ensemble played very similarly, and when they were more variable. Our research on small ensembles has found more variablity in moments that are more emotionally expressive. Is this the same for an orchestra, with more people to coordinate?\item \textbf{Inspiration alignment}: When do musicians from different sections breathe together? Entries, accents, section changes,... something in the music is prompting this coordination, even in string players and percussionists.\item \textbf{Rhythmicity and Synchrony with the Conductor}: Conducting gestures change with the rhythmic character of the music, does this translate to different qualities of synchrony across the ensemble? We expect that synchronization is more precise when the music is very rhythmical, but let's see what the data says.
\end{itemize}

 Below are some previews of emergent patterns.
\ % -----
	\SepRule
