\section*{Individual Participant}
 PC707 wore the Equivital monitor vest and the Nansense Mocap suit while performing the 2023 Lydo concerts,  as the Conductor. This section presents some views of their physiological states during these performances.

In an experiment of this scale, with many technologies and devices recordings over several days outside of laboratory conditions, there is always some data loss.  Malfunctioning sensors and changes in conditions compromised some portion of measurements from a third of performer participants, resulting in a loss of $9\%$ of respiration and $16\%$ of cardiac measurements, less than expected under the circumstances.  Fortunately, this participant’s recordings were consistently of good quality, with only occassional interference from motion.


The compound figure above shows 5 second snapshots of sensor measurements when this performer was at rest (while the MC spoke to the audience), when they were playing, and just after playing while the audience applauded. ' ' After aligning signals with recordings of the performances and extracting various features, the researchers inspect these base measurements for patterns between conditions, between performances, and between musicians. 

\begin{figure}[h]
\begin{center}
\includegraphics[width=1.000000\linewidth]{/Users/finn/Desktop/Current_Projects/Stavanger/Summaries/Plots/BPM/Cardiac_PC707_Full_BPM.jpg}
\caption{PC707 Heart beats and local heart rate during all captured performances, from the dress rehearsal (RD) at the top through the two days of school concerts (C1-C2, C3-C4) and the final family concert of the Lydo series (C5).  Each plot indicates when the orchestra was playing (beige) or listening (blue), with the beginings of each piece labeled on the x-axis.  The black line reports the median heart rate over a 60-beat interval, centred, and red dots mark the inferred heart rate from each measured beat. While these sensors are generally quite good, movement can introduce artifacts, and some of the scattering of red dots may be sensor errors. The conductor had the widest spread of heart rates during these performances. The first concert (C1) has some instability in the first piece but this seems to have resolved rapidly.}
\label{BPM}
\end{center}
\end{figure}
\begin{figure}[h]
\begin{center}
\includegraphics[width=0.800000\linewidth]{/Users/finn/Desktop/Current_Projects/Stavanger/Summaries/Plots/Card_params/Cardiac_PC707_C5__BPM_dist.jpg}
\caption{To give a sense of how listening and performance differ, these plots show the joint-distributions of Heart Rate (HR) and Heart Rate Variability (HRV) during the times (Left) that the MC was Speaking vs (Right) when the Orchestra was playing in one whole concert.  Brightness indicates the concentration of moments with specific combinations of HR and HRV values, highlighting common cardiac states.  Heart Rate Variability (HRV) is most often studied in people being still, so the change in distribution while this performer was actively performing is scientifically interesting and will be investigated more closely. The shift to less heart rate variability with higher heart rates seems to be quite common during music playing.}
\label{HRV}
\end{center}
\end{figure}
\begin{figure}[h]
\begin{center}
\includegraphics[width=0.800000\linewidth]{/Users/finn/Desktop/Current_Projects/Stavanger/Summaries/Plots/Resp_Parms/Insp_PC707_C1__dists.jpg}
\caption{To compare breathing during performance and listening, these plots show the joint-distributions of Inspiration Depth (Relative Depth) and Inspiration Duration (Inspiration time) of the performer's measured breaths during the times (Left) that the MC was Speaking vs (Right) when the Orchestra was playing over one whole concert.  Inspirations are often more slow, shallow, and variable during listening, while respiration shape during performance depends closely how they performed. Inspirations during performance varied widely, though were often faster than breaths during the MC's speechs. }
\label{resp}
\end{center}
\end{figure}
\begin{figure}[h]
\begin{center}
\includegraphics[width=0.850000\linewidth]{/Users/finn/Desktop/Current_Projects/Stavanger/Summaries/Plots/Waves/Resp_PC707_Tcha_waves.jpg}
\caption{Above are this performer's measured respiration waves during each performance of a single piece, showing the different types of breaths used in performance and how they (roughly) align.  These measurements are scaled to their most common inspiration depth while listening. Active breaths for coordination or physical exertion are often much deeper, while performance breaths for winds are deeper still. There are interesting consistencies here between performances, both in the timing of some breaths and in their changing rate and depth.}
\label{Waves}
\end{center}
\end{figure}
