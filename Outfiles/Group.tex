Analysis of these measures across performers is ongoing, with many patterns to be discovered. The orchestra is an exceptional example of human coordination, with many bodies simultaneously producing sound with great control and replicability. In the next months, we will carry out different analyses on the performers' data that will show us how they coordinate with the others in the orchestra and how musicians' bodies engage with the music. For example: 
\begin{itemize}
\item \textbf{Performance variability}: We will look at when in the concerts the ensemble played very similarly, and when they were more variable. Our research on small ensembles has found more variablity in moments that are more emotionally expressive. Is this the same for an orchestra, with more people to coordinate?\item \textbf{Inspiration alignment}: When do musicians from different sections breathe together? Entries, accents, section changes,... something in the music is prompting this coordination, even in string players and percussionists.\item \textbf{Rhythmicity and Synchrony with the Conductor}: Conducting gestures change with the rhythmic character of the music, does this translate to different qualities of synchrony across the ensemble? We expect that synchronization is more precise when the music is very rhythmical, but let's see what the data says.
\end{itemize}

 Below are some previews of emergent patterns.
\ % -----
	\SepRule
\begin{figure}[h]
\begin{center}
\includegraphics[width=0.600000\linewidth]{/Users/finn/Desktop/Current_Projects/Stavanger/Summaries/Plots/Group/Synch_taps_C2_Demo.png}
\caption{This figure shows the quantity of motion measured from each participant's physiology sensor in a specific interval of time, ordered by section. Here it is zoomed into the 30 seconds around the synchronisation taps performed by the orchestra during the second Lydo concert, showing the original device time alignment above and the effect of correcting those timings by aligning the taps. (The beginning of Clapping music was also used as a secondary check on clock alignment.) Thank you for tapping along to the beeps! This essential task allows us to study your collective coordination with much greater temporal precision.}
\label{Synch}
\end{center}
\end{figure}
\begin{figure}[h]
\begin{center}
\includegraphics[width=1.000000\linewidth]{/Users/finn/Desktop/Current_Projects/Stavanger/Summaries/Plots/Group/OrchMotion_each_performance_Stra_set.png}
\caption{The concurrent body sway across performers during 6 performances of Strauss's Radzetsky March: the dress rehearsal (RD) at the top through the two days of school concerts (C1-C2, C3-C4) and the final family concert of the Lydo series (C5).  Besides capturing the intensity and metricality of the music, these plots show differences between sections, with the lower strings working hard, and great consistency between performances. These measurements of motion will be combined with cardiac measurements, audio, and information from the scores to see how they interact.}
\label{QoM}
\end{center}
\end{figure}
\begin{figure}[h]
\begin{center}
\includegraphics[width=1.000000\linewidth]{/Users/finn/Desktop/Current_Projects/Stavanger/Summaries/Plots/Group/OrchResp_each_performance_Full_set.png}
\caption{This figure shows aligned respiration across the orchestra during the performance of one piece, Kjempevise-slåtten.  Yellow marks chest expansion, inspirations, while dark blue shows sharp expirations. The shifting texture of inspirations reflects this piece's escalation of intensity and the gradual accumulation of players. While musically aligned inspirations are expected across the winds, there are also many moments of strongly aligned inspirations across strings sections. This will be a special focus of future analysis.}
\label{Resp}
\end{center}
\end{figure}
